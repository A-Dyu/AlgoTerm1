\documentclass{article}
\usepackage[russian]{babel}
\usepackage{ragged2e}

\begin{document}
\begin{flushleft}
\section{}

а) В худшем случае increment совершит k операций (при всех a[i] = 1).

b) Рассмотрим последовательность операций increment после которой массив заполнится 1. increment будет использована $2^k$ раз. Для каждого a[i] верно что increment произведет операции с этим элементом $2^{(k - i)}$ раз $\Rightarrow$ всего будет произведено $\sum\limits_{i = 0}^{k}{2^i} = O(2^k) \Rightarrow$ за $2^k$ вызовов increment совершит $O(2^k)$ операций $\Rightarrow$ его среднее время работы $O(1)$

c) Если при всех $a[i] = 0$ вызвать decrement он совершит $O(k)$ операций и заполнит массив 1. Если при всех $a[i] = 1$ вызвать increment он выполнит $O(k)$ операций и заполнит массив 0. Если поочередно вызывать эти функции $n$ раз время работы будет $\Theta(nk)$.

d)Будем хранить индекс j, указывающий на максимальный индекс i, такой что a[i] = 1. Тогда операция setZero будет выполнять не больше действий, чем было вызванно операций increment после ее последнего вызова $\Rightarrow$ суммарное время работы всех вызовов операции setZero будет $\le$ чем у операции increment, работающей в амортизировано за $O(1) \Rightarrow$ setZero также будет работать амортизировано за $O(1)$.

\section{}

Пусть $\Phi_i = abs(\frac{m_i}{2} - k_i)$, где $m_i$ - количество выделенной памяти, а $k_i$ - количество элементов в стеке. Рассмотрим операции: 

1)push/pop (без изменения массива): $1 + \Phi_i - \Phi_{i - 1} = 1 \pm 2 = O(1)$

2)push (с расширением массива): $k + \Phi_i - \Phi_{i - 1} = k - 2 \cdot \frac{k}{2} = O(1)$

3)pop (c уменьшением массива): $m_{i - 1} / 2 + \Phi_i - \Phi_{i - 1} = m_{i - 1} / 2 - 2 \cdot m_{i - 1} / 4 = O(1)$

\section{}

Заведем массивы $b$ и $c$ размера $n$. Операция set будет выполнять $a[i] = b[i] = c[i] = x$. Операция get вернет значение в $a[i]$ при $a[i] == b[i] == c[i]$, иначе вернет 0 (Вероятность возвращения ложного значения отличного от нуля - $\frac{1}{x^2}$ (где x - количество возможных значений элементов) $\Rightarrow$ можем ей принебречь). 

\section{}

Будем хранить указатель *p, указывающий на свободный блок памяти, а также стек указателей на блоки памяти, освобожденные операцией free. Операция malloc при вызове проверяет наличие элементов в стеке. Если стек пуст, она возвращает *p, после чего совершает присвоение $*p = *p + blocksize$. Если стек хранит указатель на освобожденный блок, операция удалит его из стека и вернет пользователю. Взятие из стека и переприсвоение указателя происходят за $O(1) \Rightarrow$ malloc работает за $O(1)$. Операция free добавляет элемент в стек и расшираяет его при необходимости, эти операции амортизировано работают за $O(1)$ (см. Задание 2) $\Rightarrow$ free работает за $O(1)$. Количество выделенной памяти $O(k)$, где k - количество вызовов free $\Rightarrow$ один вызов free аммортизированно использует $O(1)$ доп. памяти

\section{}

Пусть $a$ - исходный массив. Заведем массив пар $b$ длины $k$. Проинициализируем массив парами $(\#, 0)$ (где \# - элемент, не встречающийся в массиве, а 0 - "счетчик"). Далее, для всех $a[i]$ от 0 до $n - 1$ выполним алгоритм:


1) Если в массиве $b$ нет пары с элементом $a[i]$ тогда: a) Если счетчик $b[0]$ равен 0, заменим $b[0]$ на (a[i], 0). б) Иначе отнимем от счетчика $b[0]$ единицу.


2) Если $a[i]$ есть в $b$ то добавим к его счетику 1 и переместим в массиве так, чтобы массив $b$ остался отсортирован по вторым элементам.

После этого, для каждого $a[i]$ в массиве $b$ посчитаем, сколько раз он встречается в массиве, если это число $> \frac{n}{k}$ то запишем его в ответ.

\section{}

1)add x

Пусть $\Phi_i = n \cdot \log{n} - \sum\limits_{i = 0}^{n - 1}{dist(i)}$, где $dist(i)$ - номер массива $a_j$, в котором лежит элемент $i$. Разделим операцию add на добавление элемента в массив $a[0]$ и пермещение элементов в заданном порядке и оценим их время работы:

а)Добавление: $1 + \Phi_i - \Phi_{i - 1} = 1 + \log{n} = O(\log{n})$

б)Перемещение: Совмещение массивов проходит за $O(|a| + |b|) \Rightarrow$ на перемещение одного элемента в соседний массив нужно $O(1)$ действий  $\Rightarrow \sum\limits_{j}{dist(j)_i - dist(j)_{i - 1}} + \Phi_i - \Phi_{i - 1} = \sum\limits_{j}{dist(j)_i - dist(j)_{i - 1}} - \sum\limits_{j}{dist(j)_i - dist(j)_{i - 1}} = 0 = O(1)$.

$O(1) + O(\log{n}) = O(\log{n})$


2)contains x

В худшем случае операция работает за $\sum\limits_{i = 1}^{i = \log{n}}{i} = O(\log^2{n})$. Среднее время работы операции: $\frac{\sum\limits_{i = 1}^{i = \log{n}}{\sum\limits_{j = 1}^{j = i}{j}}}{\log{n}} = O(\log{n})$.
\end{flushleft}
\end{document}