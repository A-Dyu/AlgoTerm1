\documentclass{article}
\usepackage[russian]{babel}
\usepackage{ragged2e}

\begin{document}
\begin{flushleft}
\section{}

при $ c = \frac{1}{1000}$,  
$\frac{n^3}{6} - 7n^2 \ge \frac{n^3}{1000}$ $\forall n \ge 43 \Rightarrow \frac{n^3}{6} - 7n^2 = \Omega(n^3)$

\section{}

1, $(\frac{3}{2})^2$,  $n^{\frac{1}{\log{n}}}$, $\log{\log{n}}$, $\sqrt{\log{n}}$, ${\log^2{n}}$, $(\sqrt{2})^{\log{n}}$, n, $2^{\log{n}}$, $n\log{n}$, $\log{n!}$, $n^2$, $4^{\log{n}}$, $n^3$, $n^{\log{\log{n}}}$, $(\log{n})!$, $(\log{n})^{\log{n}}$, $e^n$, $n \cdot 2^n$, $(n + 1)!$, $2^{2^n}$, $2^{2^{n + 1}}$

\section{}

$\log{n!} = \sum\limits_{i = 1}^n{\log{i}} \Rightarrow \log{n!} \le n \log{n} \Rightarrow \log{n!} = O(n \log{n})$	

$\log{n} - \log{\frac{n}{2}} = 1 \Rightarrow \sum\limits_{i = \frac{n}{2}}^n \log{i} \ge n \log{n} - \frac{n}{2}$, а $\sum\limits_{i = 1}^\frac{n}{2}\log{i} \ge \frac{n}{2} \Rightarrow$ при $c = \frac{1}{3}$ $\sum\limits_{i = 1}^n{\log{i}} \ge c \cdot n \log{n}\Rightarrow \log{n!} =  \Omega(n \log{n})$

Т.к $\log{n!} = O(n \log{n})$ и $\log{n!} = \Omega(n \log{n} \Rightarrow \log{n!} = \Theta(n \log{n})$

\section{}

$f(n) = \sum\limits_{i = 0}^{\log{n} - 1}{2^i \cdot (\frac{2^{\log{n} - i}}{\log{n} - i}}) = \sum\limits_{i = 0}^{\log{n} - 1}{\frac{n}{\log{n} - i}} = n \cdot \sum\limits_{i = 1}^{\log{n}}{\frac{1}{i}} \Rightarrow$ т.к $\sum\limits_{i = 1}^{\log{n}}{\frac{1}{i}} = \Theta(\log\log{n})$ (см. Сумма гармонического ряда) $\Rightarrow n \cdot \sum\limits_{i = 1}^{\log{n}}{\frac{1}{i}} = \Theta(n \log\log{n})$ 
\section{}

\subsection{}

Рассмотрим время работы одной итерации алгоритма, вызванного от чисел длины n. Все функции работают за $O(n)$, т.к все аргументы $ \le n$, а длина результата $\le 2 * n$. Также производится 4 рекурсивных вызова от чисел длинны $\frac{n}{2}$ ($n - k$ при $k = \frac{n}{2}$) $\Rightarrow T(n) = 4 \cdot T(\frac{n}{2}) + O(n) \Rightarrow$ По мастер теореме алгоритм работает за $O(n^{\log_2{4}}) = O(n^2)$.

\subsection{}

Время работы функций второго алгоритма останется $O(n)$, т.к все функции также вызываются от чисел длины $\le n$, функция делает 3 рекурсивных вызова от $\frac{n}{2}$ (вызов от $a + d$ и $c + d$ также $\frac{n}{2}$ т.к длина числа изменяется не больше, чем на $1) \Rightarrow T(n) = 3 \cdot T(\frac{n}{2}) + O(n) \Rightarrow$ по мастер теореме получим $O(n^{\log_2{3}})$
\end{flushleft}
\end{document}