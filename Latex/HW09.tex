\documentclass{article}
\usepackage[russian]{babel}
\usepackage{ragged2e}

\begin{document}
\begin{flushleft}
\section{}

Определим $dp(i)$ как количество уникальных подпоследовательностей, последний элемент которых $i$-ый (если существует последовательность $p_j = p_i : j < i$ последовательность $p_i$ - не уникальна). Пусть $dp(0) = 1$ (пустая последовательность). Определим $pref(i) = \sum\limits_{j = 0}^i{dp(j)}$.

 $last(i)$ - индекс предыдущего вхождения $a[i]$ (если $a[i]$ - первое вхождение $last(i) = 0$). Тогда $dp(i) = \sum\limits_{j = last(i)}^{i - 1}{dp(j)} = pref(i - 1) - pref(last(i) - 1)$ (Будем считать, что $pref(-1) = 0$). Докажем: 

1) Если добавить $a[i]$ к любой уникальной последовательности заканчивающейся на элементы с $last(i)$ по $i - 1$ то получим новую уникальную последовательность, т.к $i$ - первое вхождение $a[i]$ после $last(i) \Rightarrow$ такой последовательности не могло возникнуть на отрезке $last(i)... i - 1$. Если бы такая последовательность существовала на отрезке $1.. last(i) - 1$, то последовательности посчитанные в $last(i)... i - 1$ также бы были посчитаны на отрезке $1.. last(i) - 1 \Rightarrow$ не были бы уникальными $\Rightarrow$ противоречие.

2) Любая последовательность полученная добавлением $a[i]$ к последовательностям $1.. last(i) - 1$ будет неуникальной т.к уже была посчитана в $last(i)$

После подсчета $dp(i)$ легко получить $pref(i) = pref(i - 1) + dp(i)$

Итоговый ответ будет равен $\sum\limits_{i = 1}^n{dp(i)} = pref(n) - pref(0) = pref(n) - 1$

\section{}

1) Пусть $b$ - искомый максимальный подпалиндром. Тогда в $a^r$ присутсвует подпоследовательность $b^r$, причем $b = b^r$ т.к $b$ - палиндром $\Rightarrow$ длина максимальной общей подпоследовательности $s$ $a$ и $a^r$ $len(s)\ge len(b)$. 

2) Заметим что если существует общая подпоследовательность $s$, то в $a$ есть подпоследовательность $s$ и $s^r$ (если в $a$ нет $s^r$, то в $a^r$ нет $s$). Докажем, что если в $a$ лежат последовательности $s$ и $s^r$ то существует подпалиндром $b : len(b) = len(s)$. Рассмотрим индексы вхождений $s$ и $s^r$ слева направо, возьмем первые $\frac{len(s)}{2}$ элементов той последовательности, $\frac{len(s)}{2}$-ый элемент которой находится левее. После возьмем правые $\frac{len(s)}{2}$ элементов другой подпоследовательности и получим подпалиндром длины $len(s)$. $\Rightarrow$ для любой общей подпоследовательности $s$ $a$ и $a^r$  $\exists b : len(b) = len(s) \Rightarrow len(b) \ge len(s)$

3) Из пунктов 1 и 2 следует: $len(b) \ge len(s) \&\& len(s) \ge len(b) \Rightarrow len(b) = len(s)$ ч.т.д.

\section{}




\end{flushleft}
\end{document}